%Je nach dem in welcher Sprache ihr euer Paper schreiben wollt, benutzt bitte entweder den Deutschen-Titel oder den Englischen (einfach aus- bzw. einkommentieren mittels '%')

%Deutsch
\section{Stand des Wissens}

% Englisch
%\section{Related Work}

Ziel dieses Kapitels ist es, dem Leser zu zeigen, welche anderen verwandten Arbeiten im selben Universum / Forschungsfeld bereits Ergebnisse geliefert haben. Hier geht es darum, dass man zeigt, dass man das eigene Forschungsfeld kennt und die wichtigsten Arbeiten darin kurz zusammengefasst beschreibt. 

Genauso ist es wichtig, dass man folgenden Unterschied bzw. das DELTA herausarbeitet:
\begin{itemize}
	\item[] Your Work vs. Related Work
	\item[] ODER
	\item[] Was haben die anderen Arbeiten eben NICHT gemacht und darum möchten wir es in diesem Position Paper motivieren
\end{itemize}

Dieses Kapitel ist sehr wichtig, weil es aus der Einleitung und der Problembeschreibung heraus nochmal zeigt, dass es einen NEED gibt, um dieses Problem zu lösen (weil es bis jetzt kein anderer getan hat).

Folgende Key-Points gibt es bei diesem Kapitel zu beachten:
\begin{itemize}
	\item mind. 10 wissenschaftliche Quellen (andere Papers oder facheinschlägige Bücher) sollen identifiziert und zusammengefasst dargestellt / beschrieben werden. Dabei sollen diese wissenschaftlichen Quellen zitiert werden
	\item die wissenschaftlichen Quellen sollen sich zu den nicht wissenschaftlichen Quellen die Waage halten im Verhältnis von 2/3 wissenschaftliche Quellen zu 1/3 andere Quellen 
	\item folgende Datenbanken gelten als "gültige Datenbanken" um wissenschaftliche Quellen zu suchen:
	\subitem ACM Digital Library: \href{https://dl.acm.org/}{https://dl.acm.org/}	
	\subitem IEEE Explore: \href{https://ieeexplore.ieee.org/Xplore/home.jsp}{https://ieeexplore.ieee.org/Xplore/home.jsp} 
	\subitem Scopus: \href{https://www.scopus.com/search/form.uri?display=basic\#basic}{https://www.scopus.com/search/form.uri?display=basic\#basic}
	\subitem Science Direct: \href{https://www.sciencedirect.com/search}{https://www.sciencedirect.com/search} 
	\subitem Springer Verlag: \href{https://link.springer.com/}{https://link.springer.com/}
	\subitem Scitepress Digital Library: \href{https://www.scitepress.org/AdvancedSearch.aspx?SearchCriteria=Papers}{https://www.scitepress.org/AdvancedSearch.aspx?SearchCriteria=Papers} 
\end{itemize}

\textbf{VORGABE}: dieses Kapitel soll genau eine A4-Seite benötigen