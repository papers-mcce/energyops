%Je nach dem in welcher Sprache ihr euer Paper schreiben wollt, benutzt bitte entweder den Deutschen-Titel oder den Englischen (einfach aus- bzw. einkommentieren mittels '%')

%Deutsch
\section{Schlussfolgerungen und Ausblick}


%Englisch
%\section{Conclusion and Future Work}

Mit dem letzten Kapitel schließt das Position Paper ab und sollte im Idealfall folgende Informationen beinhalten:
\begin{itemize}
	\item Erinnert den Leser / die Leserin in welchem Universum / Forschungsfeld das Position Paper ist
	\item Erinnert den Leser / die Leserin was die Problemstellung in diesem Universum / Forschungsfeld
	\item Erinnert den Leser / die Leserin an die vorgeschlagene Herangehensweise aus Kapitel 3, wie ihr vor habt diese Problemstellung zu bearbeiten
	\item gebt dem Leser / der Leserin die wichtigesten Key-Points eures Papers mit (bevor sie aufhören zu lesen), so dass sie sich "ewig" daran erinnern bzw. eure Ideen weitererzählen
	\item gebt dem Leser / der Leserin einen Ausblick (Future Work) was ihr als nächstes tun werdet (einen Vorgeschmack auf das Final Paper)
\end{itemize}

\vspace{10pt}
\textbf{VORGABE}: dieses Kapitel soll nicht mehr als eine A4-Seite benötigen

\vspace{10pt}
Hier noch ein paar Position Paper (Beispiel) and denen ihr euch orientieren könnt:
\begin{itemize}
	\item\textbf{\textit{Paper \cite{ref01}}:} \\
	Igor Ivki\'c, Andreas Mauthe, and Markus Tauber. (2019). 
	\textit{"Towards a Security Cost Model for Cyber-Physical Systems"}. In Proceedings of the 16th Annual Consumer Communications \& Networking Conference (CCNS). 
	IEEE.\\
	\textbf{Beschreibung:} soll als Vorlage dienen, wie ein Position Paper aufgebaut sein kann\\
	\textbf{Link:} \href{https://arxiv.org/pdf/1905.06124.pdf}{https://arxiv.org/pdf/1905.06124.pdf}\\
	\item\textbf{\textit{Paper \cite{ref02}}:} \\
	Igor Ivki\'c, Harald Pichler, Mario Zsilak, Andreas Mauthe, and Markus Tauber. (2019). 
	\textit{“A Framework for Measuring the Costs of Security at Runtime”}. 
	In Proceedings of the 9th International Conference on Cloud Computing and Services Science (CLOSER).\\
	\textbf{Beschreibung:} soll als Vorlage dienen, wie ein Position Paper aufgebaut sein kann. \\
	\textbf{Link:} \href{https://arxiv.org/pdf/1905.11180.pdf}{https://arxiv.org/pdf/1905.11180.pdf}\\
	
	\item\textbf{\textit{Paper \cite{ref03}}:} \\
	Igor Ivki\'c, Patrizia Sailer, Antonios Gouglidis, Andreas Mauthe, and Markus Tauber. (2021).
	\textit{"A Security Cost Modelling Framework for Cyber- Physical Systems"}.
	ACM Transactions on Internet Technology (TOIT).\\
	\textbf{Beschreibung:} soll als Vorlage dienen, wie man ein Related Work Kapitel aufziehen kann. Dieses Paper ist ein Journal-Artikel und untersucht ein Thema in der Tiefe. Im Vergleich dazu soll bei einem Position Paper ein neues Thema motiviert werden und der Anreiz geschaffen werden, dass es wichtig wäre, dies genauer zu untersuchen. \\
	\textbf{Link:} \href{https://arxiv.org/pdf/2107.07784.pdf}{https://arxiv.org/pdf/2107.07784.pdf}\\
\end{itemize}

\newpage
Hier ist noch ein Beispiel, wie man eine Grafik einfügt:
\begin{figure}[H]
	\centering
	\includegraphics[width=10cm]{fig/Fig1.png}
	\caption{Master Cloud Computing Engineering (MCCE).}
	\Description{MCCE.}
	\label{fig:Fig1}
\end{figure}