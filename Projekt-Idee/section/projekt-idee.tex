\section{Einleitung und Problemhintergrund}

Bei der ersten Aufgabenstellung sollen die Studierenden ihre Projekt-Idee in Form eines "One Pagers" (genau eine A4-Seite) beschreiben. Dabei sollen zumindest folgende Informationen / Inhalte im "One Pager" beschrieben werden:
\begin{itemize}
	\item\textbf{1. Absatz (Universum / Forschungsfeld): } \\
	Im ersten Absatz soll das Univsersum / Forschungsfeld beschrieben werden, in dem sich die Projektarbeit befindet. Der Leser / die Leserin soll aus dem "großen Ganzen" an die eigentliche Problemstellung herangeführt werden und wissen, in welchem Themenfeld sich diese befindet. \\
	
	\textbf{\textit{ACHTUNG}}: Es ist in Ordnung, wenn die Studierenden das Universum / Forschungsfeld in mehr als nur einem Absatz herleiten. Es ist nur wichtig, dass folgende Regel eingehalten wird: ein Absatz beteht zumindest aus 3 Sätzen, wobei ein Satz aus maximal 30 Worten besteht.\\
	
	\item\textbf{2. Absatz (Problemstellung im Universum / Forschungsfeld): } \\
	Im zweiten Teil wird dem Leser / der Leserin das Problem, welche im beschriebenen Universum / Forschungsfeld besteht, aufgezeigt bzw. beschrieben. Die Problemstellung gilt dabei als die Grundlage für die Projekt-Arbeit, die es im Zuge des 2. Semesters zu lösen gilt. Die Lösung soll dabei mittels einer technischen Umsetzung (z.B.: einem Prototypen bzw. einer Demo) demonstriert werden. \\
	
	\textbf{\textit{ACHTUNG}}: Es ist in Ordnung, wenn die Studierenden die Problemstellung in mehr als nur einem Absatz herleiten. Es ist nur wichtig, dass folgende Regel eingehalten wird: ein Absatz beteht zumindest aus 3 Sätzen, wobei ein Satz aus maximal 30 Worten besteht.\\
	
	\item\textbf{3. Absatz (Lösungsansatz, um die Problemstellung zu bearbeiten): } \\
	Im letzten Teil soll eine Idee bzw. ein möglicher Lösungsansatz präsentiert und beschrieben werden, mit dem man die beschriebene Problemstellung bearbeiten möchte. Dabei ist es wichtig, dass der beschriebene Lösungsweg realistisch bzw. umsetzbar und auch nachvollziehbar ist. Es soll ein möglicher Weg sein, den man gehen kann, um die Problemstellung zu bearbieten oder sich zumindest einer Lösung anzunähern. \\
	
	\textbf{\textit{ACHTUNG}}: Es ist in Ordnung, wenn die Studierenden den Lösungsansatz in mehr als nur einem Absatz herleiten. Es ist nur wichtig, dass folgende Regel eingehalten wird: ein Absatz beteht zumindest aus 3 Sätzen, wobei ein Satz aus maximal 30 Worten besteht.
\end{itemize}