%Je nach dem in welcher Sprache ihr euer Paper schreiben wollt, benutzt bitte entweder den Deutschen-Titel oder den Englischen (einfach aus- bzw. einkommentieren mittels '%')

%Deutsch
%\section{Schlussfolgerungen und Ausblick}


%Englisch
\section{Conclusion and Future Work}
\label{conclusion:conclusion}
This paper addressed the critical challenge of energy management for SMEs operating on-premise server infrastructure through a cloud-based monitoring and optimization solution. The implementation, detailed in Section~\ref{methodology:prototype} and Appendix~\ref{appendix:github-docs}, successfully integrated IoT-based power monitoring, smart meter data, and real-time electricity pricing to provide actionable insights for cost-effective server operations.

The key contributions of this work include:
\begin{itemize}
    \item A scalable, serverless architecture combining IoT monitoring and real-time pricing data,
    as demonstrated in Section~\ref{methodology:prototype}
    \item Empirical evidence from Section~\ref{results:energy-analysis} demonstrating 
    potential cost savings of up to 10.5 cent per kWh for peak loads (CPU stress test) through intelligent workload scheduling
    \item A practical framework for SMEs to optimize server operations based on energy costs, with implementation details provided in Appendix~\ref{appendix:github-docs} (see also Appendix~\ref{appendix:quicksight-dashboard} for dashboard configuration)
\end{itemize}

A summary of the experimental results can be found in Section~\ref{results:results}.

While the evaluation demonstrated significant potential for cost optimization (see Section~\ref{results:optimization}), several limitations should be noted:
\begin{itemize}
    \item Single server configuration testing, as described in Section~\ref{methodology:test-infrastructure-setup}
    \item Limited measurement period detailed in Section~\ref{results:limitations}
    \item Focus on Austrian electricity markets (see Appendix~\ref{appendix:strompreis-api})
    \item Manual intervention requirements for workload scheduling
\end{itemize}

Future work should explore:
\begin{itemize}
    \item Machine learning for automated workload scheduling, building on the energy analysis framework (Section~\ref{results:energy-analysis})
    \item Multi-server environment support, extending beyond the current setup (Section~\ref{methodology:test-infrastructure-setup})
    \item Integration with renewable energy sources, complementing the price-based optimization (Section~\ref{results:optimization})
\end{itemize}

The results establish that energy-aware server management can provide substantial benefits for SMEs, creating a foundation for both cost reduction and environmental responsibility. As energy prices continue to rise, such integrated approaches will become increasingly critical for SME competitiveness. The complete implementation and analysis tools are available in the public repository (Appendix~\ref{appendix:github-docs}), enabling further research and practical applications in this domain.