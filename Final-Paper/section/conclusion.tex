%Je nach dem in welcher Sprache ihr euer Paper schreiben wollt, benutzt bitte entweder den Deutschen-Titel oder den Englischen (einfach aus- bzw. einkommentieren mittels '%')

%Deutsch
%\section{Schlussfolgerungen und Ausblick}


%Englisch
\section{Conclusion and Future Work}
This paper addressed the critical challenge of energy management for SMEs operating on-premise server infrastructure through a cloud-based monitoring and optimization solution. Our implementation successfully integrated IoT-based power monitoring, smart meter data, and real-time electricity pricing to provide actionable insights for cost-effective server operations.

The key contributions of this work include:
\begin{itemize}
    \item A scalable, serverless architecture combining IoT monitoring and real-time pricing data
    \item Empirical evidence demonstrating potential cost savings of up to 54.6\% through intelligent workload scheduling
    \item A practical framework for SMEs to optimize server operations based on energy costs
\end{itemize}

While our evaluation demonstrated significant potential for cost optimization, several limitations should be noted:
\begin{itemize}
    \item Single server configuration testing
    \item Limited measurement period
    \item Focus on Austrian electricity markets
    \item Manual intervention requirements for workload scheduling
\end{itemize}

Future work should explore:
\begin{itemize}
    \item Machine learning for automated workload scheduling
    \item Multi-server environment support
    \item Integration with renewable energy sources
    \item Blockchain-based peer-to-peer energy trading
\end{itemize}

The results establish that energy-aware server management can provide substantial benefits for SMEs, creating a foundation for both cost reduction and environmental responsibility. As energy prices continue to rise, such integrated approaches will become increasingly critical for SME competitiveness.