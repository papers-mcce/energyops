%Je nach dem in welcher Sprache ihr euer Paper schreiben wollt, benutzt bitte entweder den Deutschen-Titel oder den Englischen (einfach aus- bzw. einkommentieren mittels '%')

\section{Introduction}
\label{introduction:introduction}
The ongoing rise in energy prices presents a significant challenge for small and medium-sized enterprises (SMEs). Energy efficiency and cost optimization are critical concerns, especially for those operating on-premise server infrastructures. Electricity costs have become a major factor in IT operations, as servers are essential for business continuity but also represent a substantial and variable energy expense.

Servers cannot function without power, and their energy consumption fluctuates depending on workload and operational state. For example, a server under heavy computational load or during data-intensive tasks will draw more power than one in an idle or maintenance state. SMEs must perform a variety of operational and management tasks on their servers, such as reboots, backups, software updates, and system maintenance. Each of these activities can have a distinct energy profile, yet most businesses lack transparency regarding how much power is consumed by each task.

This lack of insight makes it difficult to optimize when these tasks should be scheduled, especially in the context of fluctuating electricity prices. The central question becomes: how much energy does each server activity consume, and when is the most cost-effective time to perform it?

This challenge is reminiscent of recommendations given to owners of photovoltaic (PV) systems. They are advised to run energy-intensive appliances, like washing machines, when solar production is at its peak. Similarly, SMEs could benefit from scheduling server operations during periods of lower electricity prices or when renewable energy is most available. This approach reduces costs and supports sustainability goals.

Despite the importance of these considerations, SMEs often lack integrated solutions that provide real-time visibility into server energy consumption. They also need to correlate it with specific operational tasks and align it with dynamic electricity pricing. Existing tools may monitor overall energy use but do not offer actionable insights for optimizing server management in response to market conditions, as further discussed in Section~\ref{related_work:related-work}.

To address this gap, we propose an architecture that combines smart meter data, IoT-based server power monitoring, and real-time electricity market pricing into a unified cloud-based dashboard. Our solution enables SMEs to analyze the energy impact of different server tasks, identify optimal scheduling windows, and make data-driven decisions to minimize costs. The value of this paper lies in providing a practical, scalable approach for SMEs, enhancing energy efficiency and operational flexibility in the face of rising electricity prices.

For this work, energy data from a Sagemcom smart meter will be collected via a REST Application Programming Interface (API). Additionally, data from at least one server using a NOUS A5T PowerCable will be sent to the Amazon Web Services (AWS) Cloud. The data collection process leverages AWS API Gateway for REST API connections and AWS IoT Core for Message Queuing Telemetry Transport (MQTT)-based messages. In the AWS Cloud, AWS Lambda functions process and format the incoming data before storing it in Amazon DynamoDB, which serves as our primary data repository.

The collected data is visualized in a comprehensive dashboard built with Amazon QuickSight, providing users with real-time insights into energy consumption and market prices. Another Lambda function polls the smartENERGY API to collect electricity price data from the EPEX Spot AT electricity exchange at predefined 15-minute intervals. This real-time electricity price data is also stored in DynamoDB and integrated into the QuickSight dashboard, allowing SMEs to correlate energy consumption with current market prices.

We will evaluate multiple load scenarios on the server and measure the power consumption associated with each scenario. This method reveals energy usage patterns under various operational conditions and provides valuable insights for optimization. By analyzing these results, businesses can determine the most cost-effective times to run specific workloads, minimizing energy costs in response to real-time electricity prices. 