%Je nach dem in welcher Sprache ihr euer Paper schreiben wollt, benutzt bitte entweder den Deutschen-Titel oder den Englischen (einfach aus- bzw. einkommentieren mittels '%')

\begin{abstract}
%Deutsch
%\textbf{ABSTRACT}

%Englisch
\textbf{ABSTRACT}
Small and medium-sized enterprises (SMEs) increasingly face challenges in managing rising energy costs, particularly when operating on-premise server infrastructures. This paper presents a scalable, serverless energy monitoring system tailored to the needs of SMEs, integrating smart meter data, Internet of Things (IoT)-based power monitoring, and real-time electricity market pricing into a unified cloud-based dashboard. Leveraging Amazon Web Services (AWS) serverless technologies, the system collects data from Sagemcom smart meters via the Energylive API and monitors individual server power consumption using NOUS A5T PowerCable devices, while incorporating European Power Exchange (EPEX) Spot market prices.
We detail a measurement methodology for comprehensive energy profiling across diverse operational scenarios, including peak computational loads, input/output (I/O) stress tests, system reboots, maintenance, and idle states. The proposed approach enables SMEs to optimize server workload scheduling in response to dynamic energy pricing, potentially reducing operational costs by shifting energy-intensive tasks to periods of lower electricity prices.
Our results demonstrate that the system provides actionable insights into energy consumption patterns and their economic implications, offering a cost-effective solution with minimal operational overhead. This work contributes a practical framework for SMEs seeking to enhance energy efficiency and cost management in server operations.
\end{abstract}