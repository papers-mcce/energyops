%Je nach dem in welcher Sprache ihr euer Paper schreiben wollt, benutzt bitte entweder den Deutschen-Titel oder den Englischen (einfach aus- bzw. einkommentieren mittels '%')

\begin{abstract}
%Deutsch
%\textbf{ABSTRACT}

%Englisch
\textbf{ABSTRACT}

The ongoing rise in energy prices presents a significant challenge for small and medium-sized enterprises (SMEs) operating on-premise server infrastructures. This paper addresses the critical gap in energy management solutions specifically tailored for SMEs, who often lack transparent insights into their server infrastructure's power consumption patterns and struggle to respond dynamically to fluctuating electricity prices.
We propose a comprehensive architecture that integrates smart meter data, Internet of Things (IoT)-based server power monitoring, and real-time electricity market pricing into a unified cloud-based dashboard. Our solution leverages Amazon Web Services (AWS) serverless technologies to collect data from a Sagemcom smart meter via the Energylive API and individual server power consumption through NOUS A5T PowerCable devices, while simultaneously incorporating European Power Exchange (EPEX) Spot market pricing.
The proposed measurement methodology outlines an approach for detailed energy profiling across various operational scenarios, including maximum computational load, Input / Output (I/O) stress testing, system reboot cycles, maintenance operations, and idle states. This integrated approach enables SMEs to make informed decisions about server workload scheduling based on energy costs, potentially shifting energy-intensive computational tasks to periods of lower pricing.
The resulting system provides visibility into energy consumption patterns and their economic implications, offering a scalable, cost-effective solution that minimizes operational overhead. This is a critical consideration for resource-constrained SMEs.
\end{abstract}